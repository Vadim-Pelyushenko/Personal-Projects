\documentclass[12pt]{article}

\usepackage{amsmath}

\title{A Generalized Algorithm for producing Integer Power Reduction Formulas
of Cosine}

\begin{document}
\maketitle
\noindent \textbf{The idea behind this algorithm was inspired by a video 
entitled \textit{cos(1) + ... + cos(n)} by Peyam R. Tabrizian}
\\
\textbf{https://www.youtube.com/watch?v=7LBQTpiK-Xg}
\\
\\
\textbf{Prior to seeing this video, I understood that complex numbers stemmed from
the idea that $\sqrt{-1} = i$, and that complex numbers had some properties and
could be dealt with in particular ways, but I did not believe they had any utility.}
\\
\\
\textbf{After watching the video however, I became a believer. So I present a way
a way I've discovered, for generating the integer power reduction formulas for cosine which hinges on the imaginary representation of cosine.}
\newpage
\noindent This analysis begins with (1)Euler's Formula:
$e^{i\theta} = \cos(\theta) + i\sin(\theta)$
\\
Plugging in $-\theta$, we see that 
$e^{-i\theta} = \cos(-\theta) + i\sin(-\theta) = \cos(\theta) - i\sin(\theta)$
\\
\\
We can then add these two equations to get the following identity(2):
\\
$e^{i\theta} + e^{-i\theta} = 2\cos(\theta) \to 
\cos(\theta) = \frac{1}{2}(e^{i\theta} + e^{-i\theta})$
\\
\\
This will be very important, because it means that whenever we can turn, for instance, $e^{7i} + e^{-7i} \to 2\cos(7x)$
\\
\\
So lets start applying what we now know.
\begin{align}
\cos^2\theta &= \cos(\theta) \cdot \cos(\theta) 
\\
&= \frac{1}{4}(e^{i\theta} + e^{-i\theta})^2
\\
&= \frac{1}{4}(e^{2i\theta} + 2 + e^{-2i\theta})
\\
&= \frac{1}{4}(2cos(2\theta) + 2)
\\
&= \frac{cos(2\theta) + 1}{2}
\end{align}
Which is precisely the power reduction formula! (You can verify this with a search online)


\end{document}