\documentclass[12pt]{article}

\usepackage{amsmath}

\title{A Generalized Algorithm for producing Integer Power Reduction Formulas
of Cosine}

\begin{document}
\maketitle
\noindent \textbf{The idea behind this algorithm was inspired by a video 
entitled \textit{cos(1) + ... + cos(n)} by Peyam R. Tabrizian}
\\
\textbf{https://www.youtube.com/watch?v=7LBQTpiK-Xg}
\\
\\
\textbf{Prior to seeing this video, I understood that complex numbers stemmed from
the idea that $\sqrt{-1} = i$, and that complex numbers had some properties and
could be dealt with in particular ways, but I did not believe they had any utility.}
\\
\\
\textbf{After watching the video however, I became a believer. So I present a way
a way I've discovered, for generating the integer power reduction formulas for cosine which hinges on the imaginary representation of cosine.}
\newpage
\noindent This analysis begins with (1)Euler's Formula:
$e^{i\theta} = \cos(\theta) + i\sin(\theta)$
\\
Plugging in $-\theta$, we see that 
$e^{-i\theta} = \cos(-\theta) + i\sin(-\theta) = \cos(\theta) - i\sin(\theta)$
\\
\\
We can then add these two equations to get the following identity(2):
\\
$e^{i\theta} + e^{-i\theta} = 2\cos(\theta) \to 
\cos(\theta) = \frac{1}{2}(e^{i\theta} + e^{-i\theta})$
\\
\\
This will be very important, because it means that whenever we can turn, for instance, $e^{7i} + e^{-7i} \to 2\cos(7x)$
\\
\\
So lets start applying what we now know.
\begin{align}
\cos^2\theta &= \cos(\theta) \cdot \cos(\theta) 
\\
&= \frac{1}{4}(e^{i\theta} + e^{-i\theta})^2
\\
&= \frac{1}{4}(e^{2i\theta} + 2 + e^{-2i\theta})
\\
&= \frac{1}{4}(2cos(2\theta) + 2)
\\
&= \frac{cos(2\theta) + 1}{2}
\end{align}
Which is precisely the power reduction formula! (You can verify this with a search online) Lets apply this to some more powers of cosine. We'll be reusing the expanded complex form for each subsequent power.
\\
I will not prove the formulas are correct rigorously. But viewing them on
desmos they certainly appear to match up. 
\\
https://www.desmos.com/calculator/1a6havywh1
\newpage
\begin{align}
\cos^3\theta &= cos^2\theta \cdot cos\theta
\\
&= \frac{1}{8}(e^{2i\theta} + 2 + e^{-2i\theta})(e^{i\theta} + e^{-i\theta})
\\
&= \frac{1}{8}(e^{3i\theta} + 3e^{i\theta} + 3e^{-i\theta} + e^{-3i\theta})
\\
&= \frac{1}{8}(2\cos(3\theta) + 6cos(\theta))
\\
&= \frac{1}{4}\cos(3\theta) + \frac{3}{4}\cos(\theta)
\end{align}

\begin{align}
\cos^4\theta &= cos^3\theta \cdot cos\theta
\\
&= \frac{1}{16}(e^{3i\theta} + 3e^{i\theta} + 3e^{-i\theta} + e^{-3i\theta})(e^{i\theta} + e^{-i\theta})
\\
&= \frac{1}{16}(e^{4i\theta} + 4e^{2i\theta} + 6 + 4e^{-2i\theta} + e^{-4i\theta})
\\
&= \frac{1}{16}(2\cos(4\theta) + 8\cos(2\theta) + 6)
\\
&= \frac{1}{8}\cos(4\theta) + \frac{1}{2}\cos(2\theta) + \frac{3}{8}
\end{align}

\begin{align}
\cos^5\theta &= cos^4\theta \cdot cos\theta
\\
&= \frac{1}{32}(e^{4i\theta} + 4e^{2i\theta} + 6 + 4e^{-2i\theta} + e^{-4i\theta})(e^{i\theta} + e^{-i\theta})
\\
&= \frac{1}{32}(e^{5i\theta} + 5e^{3i\theta} + 10e^{i\theta} + 10e^{-i\theta} + 5e^{-3i\theta} + e^{-5i\theta})
\\
&= \frac{1}{32}(2\cos(5\theta) + 10\cos(3\theta) + 20\cos(\theta))
\\
&= \frac{1}{16}\cos(5\theta) + \frac{5}{16}\cos(3\theta) + \frac{5}{8}\cos(\theta)
\end{align}
If you were paying attention to the coefficients at (3), (8), (13), (18), you might have noticed a relationship to pascal's triangle...

\newpage

Let's draw something like pascal's triangle, where each column signifies a coefficient in the expanded complex form of $cos^nx$. 
Each row is the previous row multiplied by $(e^{i\theta} + e^{-i\theta})$.
\\
\\
\begin{tabular}{|c|c|c|c|c|c|c|c|c|c|c|c|}
\hline
$cos^n$ & $e^{-5i\theta}$ & $e^{-4i\theta}$ & $e^{-3i\theta}$ & $e^{-2i\theta}$ & $e^{-i\theta}$ & $e^0$ & $e^{i\theta}$ & $e^{i2\theta}$ & $e^{i3\theta}$ & $e^{i4\theta}$ & $e^{i5\theta}$
\\
\hline
1 &  &  &  &  &  & 1 &  &  &  &  & 
\\
\hline
$\cos(\theta)$ &  &  &  &  & 1 & 0 & 1 &  &  &  & 
\\
\hline
$\cos^2(\theta)$ &  &  &  & 1 & 0 & 2 & 0 & 1 &  &  & 
\\
\hline
$\cos^3(\theta)$ &  &  & 1 & 0 & 3 & 0 & 3 & 0 & 1 &  & 
\\
\hline
$\cos^4(\theta)$ &  & 1 & 0 & 4 & 0 & 6 & 0 & 4 & 0 & 1 & 
\\
\hline
$\cos^5(\theta)$ & 1 & 0 & 5 & 0 & 10 & 0 & 10 & 0 & 5 & 0 & 1
\\
\hline
\end{tabular}
\\
\\
This makes sense. Because if we have a term $e^{\alpha i\theta}$ in the previous row, and we multiply by $(e^{i\theta} + e^{-i\theta})$. We should have the terms $e^{(\alpha + 1) i\theta}$ and $e^{(\alpha - 1) i\theta}$ in the next iteration. If we were to be continuously multiplying by $(e^{i\theta} + 1)$ instead then we would have exactly pascal's triangle, but instead we have pascal's triangle with gaps that are 0's.
\\
\\
It's important to note that this is the step before the conversion into cosine's of multiples of $\theta$. Furthermore in the final conversion, everything must be divided by $2^n$, and all coefficients except for $e^0$ must be multiplied by 2(as the formula dictates).
\\
\\
This is enough information to make a program that writes out each power reduction formula up to $n$ iteratively...

\end{document}